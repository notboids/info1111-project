\documentclass[a4paper, 11pt]{report}
\usepackage{blindtext}
\usepackage[T1]{fontenc}
\usepackage[utf8]{inputenc}
\usepackage{titlesec}
\usepackage{fancyhdr}
\usepackage{geometry}

\usepackage[english]{babel}
\usepackage{apacite}

\geometry{ margin=30mm }
\counterwithin{subsection}{section}
\renewcommand\thesection{\arabic{section}.}
\renewcommand\thesubsection{\thesection\arabic{subsection}.}
\usepackage{tocloft}
\renewcommand{\cftchapleader}{\cftdotfill{\cftdotsep}}
\renewcommand{\cftsecleader}{\cftdotfill{\cftdotsep}}
\setlength{\cftsecindent}{2.2em}
\setlength{\cftsubsecindent}{4.2em}
\setlength{\cftsecnumwidth}{2em}
\setlength{\cftsubsecnumwidth}{2.5em}


\begin{document}
\titleformat{\section}
{\normalfont\fontsize{15}{0}\bfseries}{\thesection}{1em}{}
\titlespacing{\section}{0cm}{0.5cm}{0.15cm}
\titleformat{\subsection}
{\normalfont\fontsize{13}{0}\bfseries}{\thesubsection}{0.5em}{}
\titlespacing{\section}{0cm}{0.5cm}{0.15cm}

%=======================================================================================

\begin{titlepage}
\center 
\textbf{\huge INFO1111: Computing 1A Professionalism}\\[0.75cm]
\textbf{\huge 2022 Semester 1}\\[2cm]
\textbf{\huge Team Project Report}\\[3cm]

\textbf{\huge Submission number: 1}\\[0.75cm]
\textbf{\huge Team Members:}\\[0.75cm]
\textbf{\large
    \begin{tabular}{|p{0.5\textwidth}|p{0.3\textwidth}|p{0.2\textwidth}|}
        \hline
        Name & Student ID & Levels being attempted in this submission\\
        \hline
        Isaac Kim & 510603294 & Task 1, 2 \\
        Jake Lewitton & 510440541 & Task 1 \\
        Tan Ky le & 510565486 & Level 1, 2 \\
        \hline
    \end{tabular}
}\\[0.75cm]
\end{titlepage}

%=======================================================================================
\tableofcontents
\newpage
%=======================================================================================
\section{Level 1: Basic Skills}

\subsection{Developing industry skills}
    \subsubsection{1. Work on yearly projects}
    Every year, start by picking a medium-sized project in a new area of computing, whether that’s a new language, framework or piece of technology. Over the year, work to complete that project and try to meet your original specifications by the end of the year. This will not only help with learning new skills, but also with project management skills. This is because it will help you set a long term goal and figure out the best way to meet it. Because it goes over a year-long period, it can be said that it would mimic the workflows of building large projects in the workplace.
    \subsubsection{2. Learn new skills every month}
    At the beginning of each month pick a brand new niche in coding. This would be a niche like a new Java framework, or a specific terminal tools/command line tool, or a specific computing concept, like blockchain technology, or encryption. Once you have picked the topic, you must then create three goals. The aim of these goals is not to have you complete three things, but to have three things that you could confidently say you would be able to build. This will help organize the learning process. Then, spend 5-10 hours a week on the skill and see that over the next year, you have learned 12 things.
    \subsubsection{3. Attend tech events and conferences}
    Tech events and conferences are great chances to keep up with the newest and most innovative technology and application. Those events could be an informative source of new knowledge and insights. Through them, you can join workshops, meet and talk to your peers and listen to new update from the industry. If time or money is what prevents you from taking part in such events, some events and conferences are hosted online which is really convenient.
    \subsubsection{4. Focus on practical skills}
    It's easy to feel frustrated when you don't understand the basics of operating systems or networking. Other times, you may be having trouble comprehending trigonometric rules used in game software or big O notation of a specific algorithm. However, rather than continuing to strive to comprehend computers, I feel it is more vital to focus on what we can create and achieve with computer tools as developers. In addition, several handy bundles of code known as API that will automate a tremendous amount of work with few lines of code have been created thanks to many developers around the world. After all, no one in the twenty-first century should be writing code in 0s and 1s.
    \subsubsection{5. Work to enhance problem solving skills}
    Problem resolution is at the heart of programming. I believe that the most difficult obstacle for most programmers is attempting to address larger problems as we begin to work on individual/group projects or industry tasks. As a result, as computer science graduates, we should be constantly working to enhance our ability to break down larger difficulties into smaller components. When we build a software product, we should think like an entrepreneur and a future customer of a service, not an engineer or a designer, because this will improve our ability to understand and discover the right challenges. In addition, I would advise that we try to visualise the situation and brainstorm with new ideas rather than only following the directions and work on to raise our voices.

%\subsection{Skills: Nobody : Computer Science}
 %   This section will be kept blank because our group only has three members.

\subsection{Skills: Isaac Kim : Data Science}
    \subsubsection{1. Statistics}
    Statistics may look to first-year university students like myself to be a series of tedious calculations using fixed formulas. After all, that's what we've been doing in "Statistics" classes since high school. As a result, it's mind-boggling to grasp how statistical analysis and probability have such a profound impact on our daily lives. Statistics are used to forecast weather, restock store shelves, assess the state of the economy, and a variety of other things.\cite{isaac1} Statistics on its own is a powerful tool for getting useful insights, but when combined with computer algorithms, machine learning, or AI, it can investigate a far larger pool of data and provide answers to business, science, and societal concerns that were previously deemed unsolvable.\cite{isaac2}

    \subsubsection{2. Programming Knowldege}
    Data science necessitates a mix of math and programming skills because it sits at the intersection of analytics and engineering. Data scientists will be unable to distinguish themselves from traditional statisticians if they are unable to use programming tools. Real data scientists will use sophisticated languages like Python or R to gain access to a wide range of useful functionality for dealing with Math, Statistics, and Scientific operations.\cite{isaac3} Data scientists can utilise computer programmes to apply statistical expertise in a quick and secure manner, allowing them to uncover patterns in large amounts of disorganised data and make use of them.\cite{isaac1}

    \subsubsection{3. Data Wrangling}
    Data wrangling is the process of organising and transforming new data into the appropriate format for later analysis. It's a crucial step that's well worth the investment because it helps analysts make better data-driven judgments in less time. Outlier treatment, missing value imputation, scaling, and correcting data types can help make a data collection more orderly and aid in the development of more accurate statistical results.\cite{isaac4} Data wrangling, in my opinion, is just as significant as Oxford Dictionary's decision to structure its print dictionary alphabetically. It couldn't be worse if an English learner had to go through 1000 pages of paper simply to discover a definition of one word. 

    \subsubsection{4. Data Visualisation}
    It might be difficult to grasp what a data collection has to offer as a story when looking at a series of statistics or numerical summaries. This is where Data Visualization, the more entertaining side of data science, may be able to assist with the solution. Data visualisation enables data scientists to convey complex data sets to scholars and the general public in a more appealing and intelligible manner.\cite{isaac1} Furthermore, because computer software can automate the process, creating complex histograms or pie charts takes only a few seconds and does not require the assistance of a skilled artist.\cite{isaac1}

    \subsubsection{5. Machine Learning}
    Machine learning allows data scientists to easily uncover patterns in a variety of data sources and make continuous improvements to their models without the need for human involvement.\cite{isaac5} Google and Facebook, for example, may develop highly targeted ad recommendation algorithms as more data from various people accumulates. Similarly, a skilled analyst could use data gathered by a Google search to find and predict popular software skills. On a larger scale, reliable prediction models, such as today's weather forecast system, can be constructed.\cite{isaac1}

    \subsubsection{6. Deep learning}
    Deep learning is a step forwards in machine learning because it uses a revolutionary method to decision-making based on a learning model that analyses data with a logical structure in much the same way that a human would.\cite{isaac5} Deep learning applications use a layered framework of algorithms known as an artificial neural network to do this analysis.\cite{isaac5} Using this new, sophisticated technique, data scientists may construct software for self-driving cars that can distinguish between traffic signs, detect pedestrians, and identify automobiles. \cite{isaac1} Deep learning can also be used to anticipate protein structures, create architectural plans, and design semiconductors. I am convinced that it has limitless potential for use in a variety of industries in the future.

    \subsubsection{7. Big Data Anaylsis}
    We produce 2.5 quintillions of data each day!\cite{isaac1} The growth of the internet, social media networks, and the internet of things (IoT) has resulted in a rapid increase in the amount of data we generate. Volume, velocity, and veracity are all high in this data.\cite{isaac6} If businesses do not properly analyse their acquired data in this data-rich environment, they will miss out on chances for wiser business decisions, more effective operations, bigger profitability, and happier consumers.\cite{isaac6} This is why data analysts are so important in today's enterprises: they help organisations minimise expenses, make quicker, more rational choices, and catch client demand at the appropriate moment to generate niche goods.\cite{isaac1}

    \subsubsection{8. Story-telling Skill}
    If it weren't for storytelling and visualisation, all of the data analysis and insights we create as data scientists would be pointless. Putting numbers and facts from your analysis on the table seldom gets you far. Imagine a weather forecaster walking in to tell people about an approaching blizzard. Their warning won't have any impact on the audience if they don't use appropriate visuals and storytelling techniques.\cite{isaac7} Hence, forecasters use graphics and interactive methods to keep viewers hooked and informed. \cite{isaac7}

    \subsubsection{Teamwork}
    I believe that the most significant element of data scientists is their ability to work in groups. Because data science is such a diverse area, it necessitates collaboration among specialists from other professions. We can't be specialists in all of the subjects and abilities listed above as individuals. It is vital to concentrate on a few abilities in which we are interested and confident while also learning from our team to improve other skills.

\subsection{Skills: Tan Ky Le : Software Development}
	\subsubsection{1.Programming languages}
	This is pretty obvious, a software developer needs to build a software from nothing and improve existing software which requires knowledge and capability over different languages. For a career as a software developer, I think one should master Java, Python and C++. Python has really simple syntax among programming languages and has been used in building mobile apps and webs, Java is a high-level language and software platform and C++ is the language for operating system programming and browsers\cite{ky1}. Knowing many coding languages allows more flexibility in choosing the field to work in as well as changing to different fields.
	\subsubsection{2.Debugging and testing}
	A software developers should not only know to code but also need to know how to test the software and fix bug. A lot of time and effort of a software project is put in testing and debugging since it is crucial that the software operates as expected. Usually, a team working on a software project has a tester to test the codes, however, one should be able to test his own code and I think it is easier to fix your code if you test it yourself. \cite{ky2}
	\subsubsection{3.Algorithms and data structures}
	It is important to understand and get familiar with different algorithms and data structures. During a software developer career, one may have to create a brand-new software or improve existing ones, thus the knowledge of algorithms and data structures will come in handy. Since the runtime and space are often required to be optimized, the developer needs to know what their software prioritize and flexibly apply different data structures and algorithms. Therefore, knowledge of algorithms and data structures will definitely be a strong skill for a software developer.
	\subsubsection{4.Version control tool (Git)} 
	Git is extremely useful as it facilitates teamwork on a project and it allows version control which is really convenient to go back and forth between different versions of the project. Software development, which is mostly worked on as a team and consists of many procedures, can surely use git for it’s convenience. Therefore, one should put time and effort in learning git, get comfortable with version control notions such as branching, merging and the command lines in git bash.
	\subsubsection{5.Teamwork skills}
	 Software developers often work together as a team for better productivity as the work can be divided to people specializing in it. For example, a project manager who keeps things going as planned, a coder to write lines of code for the software to work.
As a result, one should learn to work as a group since university. Teamwork means each person or a small group work on their allocated part of the project which offers more time, more creativity and possibilities, thus being a great benefit of teamwork. Therefore, teamwork is an important skill for people working in the industry.
	\subsubsection{6.Time management}
	 To get the work done on time, one needs to manage and use time appropriately. It causes troubles if the work is not done on time, especially in a scenario where the work is a part of a bigger project and as a result can effect the progress of other people which is totally undesired. Good time management boosts productivity, efficiency and reduces stress while poor time managements impacts personal reputation, causes lowered motivation and energy and affect others. Thus, time management is an important skill for both life and work.
	\subsubsection{7.Self-learning ability}
	 This is the skill we are expected to acquire from this unit. I think for any field in the IT industry, people have to do a lot of self-learning in order to catch up with the advances of technology. Technology moves on everyday and if one decides to just stand where they are, they will be left behind and replaced by others. We cannot expect schools or university to teach us all we will ever need to last long in the industry  because a lot of knowledge is not yet to be available at the moment. Luckily, the internet has a lot of resources for one to learn from. Therefore, self-learning skill is one crucial skill for developers.
	\subsubsection{8.Logical thinking}
	 This is the skill that I think is the most important skill of all. Logical thinking is the key to create efficient software codes, apply appropriate data structures and algorithms to  optimize the software. I believe that it is the skill that differentiates good developers from others. Logical thinking can be acquired through learning and practices, often tiring but really rewarding for the long run. 

\subsection{Skills: Jake Lewitton : Cyber Security}
    \subsubsection{1.Encryption}
    Encryption involves changing data in a way that makes it unintelligible. This is with the intention of using a reverse method (decryption) to make the data intelligible once again. Data encryption is so important to cyber security because data is not truly safe in any sort of storage space, until it is encrypted. It should always be assumed that data can be stolen/leaked, so the best way to secure said data, is to have it encrypted \cite{jake1}. Data is encrypted and decrypted using a key. A key is a piece of data that, when used along with a decryption or encryption process, is able to decrypt or encrypt information.
    \subsubsection{2.Firewall Configuration}
    The purpose of a firewall is to block malicious traffic from entering a network, or computer. Despite the simple nature of a firewall - block traffic from certain places, the operations and specifics of building a firewall that works when it needs to and doesn't work when it shouldn't is really complicated. This means it is beyond important for cyber security experts to understand how firewalls work, how to conifgure them, and most importantly, diagnose when they go wrong \cite{jake2}. Understanding firewalls, despite being a basic skill, is one that will help security experts massively in the long run.
    \subsubsection{3.Antivirus/Antimalware Software}
    Malware is software that has been created or uploaded somehwere with malicious intent. This intent could be to steal data, slow operation or use computing power for DoS attacks (to name a few cases). A type of malware is a virus, hence the name antivirus. Malware comes in a lot of shapes and sizes, making developing antimalware software really tedious. Not only do cyber security experts need to know how to comabt malware, but they also need to be on top of the latest tricks and techniques used by malicious groups to develop malware \cite{jake3}.
    \subsubsection{4.Security Auditing}
    Security audits are really important to any business that develops, monitors or maintains software. An audit is an effort to assess and analyse an entities transactions in order to ascertain what position they are in. A security audit is one in which the goal is to assess the soundness of a softwares security processes. Effective security audits should assess the following \cite{jake4}:
    \begin{enumerate}
        \item Physical components of the information system
        \item Any security patches that have been released in applications or software
        \item Any threats within the network system internally or externally
        \item The most important one. How employees collect, store and communicate their information
    \end{enumerate}
    \subsubsection{5.Data Management}
    Data management is a really important, intricate and mostly tedious process that requires massive attention to detail. When collecting small amounts of data from a small sample, it can be easy to underestimate the skill required to store data, but once that sample grows from 10 to 100,000, the attitude begins to change. Data managemtn is about finding the most secure and succinct way to store data in a way that does not all any external, or even interal user of software to access any confidential information. Careful consideration must happen in order to assess whether data has been or is managed in a safe way \cite{jake5}.
    \subsubsection{6.Digital Forensics}
    Digital forensics isn't quite the job of cyber security managers, but it deserves a mention as something that cyber security experts must have knoledge or skills about. Digital forensics is the act of using data that has been stored, or traces of online activity that can be found to help solve crimes. These crimes are normally online, so they greatly effect the day-to-day operations of software companies. This work is done by large authoritative bodies, yet it still matters to cyber security experts, as they are tasked sometimes with ensuring their systems provide enough information to catch bad actors within their system \cite{jake6}.
    \subsubsection{7.Identity and Access Management}
    Right now, software systems are big. Big means millions, if not sometimes billions of users. Most of the time, each user of a software system is priviledged to a different amount/level of information about certain things. For example, an Instagram user can only see posts of their friends if they allow them to. It is the job of a security expert to ensure that the right users always have access to the right information, and the wrong users do not. Such an intricate and accurate system of access management requires constant modification and updating, as the way we use software is always updating, and the roles of users are too.
    \subsubsection{8.Attention to detail}
    While not being a hard skill, this soft skill is extremely important to any cyber security developer. Throughout any coding journey, developers will inevitably come into contact with a large piece of code that is so riddled with bugs, it almost feels like it might be better starting off from scratch. Through becoming a developer who notices these issues early on, and constantly plugs up security flaws, you are able to instantly build software faster, and software that is better than most other developers. As Steve Jobs has said, "Details matter. It's worth waiting to get it right." \cite{jake7}


%=======================================================================================

\newpage
\section{Level 2: Basic Technology}

\subsection{Netflix tech stack - MEAN:  Isaac Kim}

\subsubsection{1. MongoDB}
    MongoDB is a NoSQL database that is open source and geared for cloud applications. \cite{isaac8} MongoDB, in short, aids in the creation of robust and responsive databases. It's a modern database system that works with massive volumes of distributed data that don't fit well in a rigid, relational model. MongoDB's architecture is made up of collections and documents, rather than tables and rows, as in older relational databases. \cite{isaac9} Documents that contain a data structure made with field and value pairs, similar to JSON, are the basic units of MongoDB. \cite{isaac9}

\subsubsection{2. ExpressJS}
	Express is a lightweight framework that runs on top of Node.js' web server functionality. It helps simplify existing APIs and introduces useful additional features. Express essentially serves as a backend that manages all the interactions between frontend and the database, ensuring a smooth data transfer from server to its end users. \cite{isaac8}

\subsubsection{3. AngularJS}
Angular is a JavaScript MVC framework for building dynamic web applications on the client side. \cite{isaac10} Angular began as a Google project, but it is currently available as an open source framework. \cite{isaac8} It does not require new syntax or language because it is built purely on HTML and JavaScript. Angular took front end development to a whole new next level by transforming static html into dynamic html. It is a very popular and crucial tool in today’s web design. 

\subsubsection{4. Node.js}
Node.js is an open source JavaScript framework that processes different connections simultaneously using asynchronous events.\cite{isaac8}  It is the MEAN stack's core. It includes an integrated web server that will provide data to users using AngularJS while also deploying the MongoDB database and application to the cloud.\cite{isaac8} The scalability of Node.js is one of its strongest features, since it easily responds to use increases.

\subsubsection{5. MEAN}
MEAN is an open source web stack for building cloud-hosted apps. MEAN stack apps are scalable, adaptable, and versatile, making them ideal for cloud hosting.\cite{isaac8} The stack comes with its own web server, making it simple to install, and the database can be expanded on demand to handle temporary traffic surges. \cite{isaac8} The MEAN stack takes advantage of the cloud's cost savings and performance enhancements.\cite{isaac8} 

The above explanation of MEAN is a straightforward answer to why Netflix could be using MEAN. Netflix is responsible for about fifteen percent of all internet traffic globally. Netflix grew incredibly fast as a company. I believe one of the most crucial factors behind their success was their quick adaptation of strategies to handle overflowing user connections globally. Netflix started using AWS(Amazon Web Server) as their cloud provider to move away from fixed relational databases and head towards highly reliable and scalable distributed servers.\cite{isaac11}  MongoDB, Express, Angular, and Node.js together must have played a very important role in achieving this, enabling Netflix to respond to user requests flexibly with the power of cloud computing and quickly with usage of consistent data structures within the program.


\subsection{Tech Stack: add student 2 name here}

Your text goes here

\subsection{ Instagram Tech Stack: Tan Ky Le}
\subsubsection{1. Django}
 Django is an open-source web framework entirely written in Python which is great for social media sites. According to Instagram engineering team, Instagram utilizes Django and features the world’s largest deployment of Django web framework\cite{ky3}. Django is the core of Instagram. To report errors on Instagram, the team at Disqus wrote Sentry which is a Django backed app. Django is integrated strongly helping Instagram starts quickly\cite{ky4}. 
\subsubsection{2. Objective-C}
Objective-C can be used to create phone applications for IOS system, in this case is Instagram.
\subsubsection{3. React Native}
Instagram user app (front end) is written in React Native, a cross-functional programming language for building and developing applications on multiple platforms such as IOS, Android operating systems.\cite{ky7}. Using React Native allows Instagram to be developed cross-platform.
\subsubsection{4. Celery and RabbitMQ}
All asynchronous activities, such as delivering notifiacations and running background processes, are handled by Celery and RabbitMQ. Celery is a task queue that is built on message communication dispersion. It's priority involves real-time activities. Celery is also helpful for scheduling. RabbitMQ is a messaging platform and is highly compatible with Celery. It is written down using the Advanced Messaging Queuing Protocol (AMQP)\cite{ky4}.
\subsubsection{5. PostgreSQL and Cassandra}
Instagram mainly uses PostgreSQL and Cassandra as back-end systems. Both have mature replication frameworks that fit as a globally consistent data store\cite{ky5}. PostgreSQL is used to contain Instagram’s data such as images, videos, articles and user information\cite{ky4}. “Most of our data(users, photo meta, tags, etc..) lives in PostgreSQL” said the engineering team. Instagram uses Cassandra as key-value storage, assisting user photo feed, direct messages and fraud detection as well\cite{ky6}
\subsection{Tech Stack: add student 4 name here}

Your text goes here


%=======================================================================================

\newpage
\section{Level 3: Advanced Skills}

Level 3 focuses on more advanced technical skills (\LaTeX\ and Git) and analysis of linkages and relationships between the items in the company tech stack.

The following is a list of advanced Git and \LaTeX\ skills/features. Each student should select one pair of items from each list and demonstrate actual use of each item (either through activity in Git, or through including items in this report). (Target = $\sim$100 words per student for each feature).
\begin{itemize}
    \item Git
    \begin{itemize}
        \item Rebasing and Ignoring files
        \item Forking and Special files
        \item Resetting and Tags
        \item Reverting and Automated merges
        \item Hooks and Tags
    \end{itemize}
    \item \LaTeX\ 
    \begin{itemize}
        \item Cross-referencing and Custom commands
        \item Footnotes/margin notes and creating new environments
        \item Floating figures and editing style sheets
        \item Graphics and advanced mathematical equations
        \item Macros and hyperlinks
    \end{itemize}
\end{itemize}

\subsection{Advanced features: add student 1 name here}

Explain your use of the advanced Git and \LaTeX\ features. 

\subsection{Advanced features: add student 2 name here}

Explain your use of the advanced Git and \LaTeX\ features. 

\subsection{Advanced features: add student 3 name here}

Explain your use of the advanced Git and \LaTeX\ features. 

\subsection{Advanced features: add student 4 name here}

Explain your use of the advanced Git and \LaTeX\ features. 



%=======================================================================================

\newpage
\section{Level 4: Advanced Knowledge}

Level 4 focuses on analysing your particular tech stack and considering alternatives. Each student should consider the tech stack they described for Level 2, and then discuss each of the following points:
\begin{itemize}
    \item What are the strengths and limitations of this stack? (Target = $\sim$200 words).
    \item What alternatives exist, and under what situations might these alternatives be a better choice? (Target = $\sim$200 words).
\end{itemize}

\subsection{Advanced Knowledge: add student 1 name here}

Your text goes here

\subsection{Advanced Knowledge: add student 2 name here}

Your text goes here

\subsection{Advanced Knowledge: add student 3 name here}

Your text goes here

\subsection{Advanced Knowledge: add student 4 name here}

Your text goes here



%=======================================================================================

\newpage

\bibliographystyle{apacite}
\bibliography{main}

\end{document}
\end{report}
