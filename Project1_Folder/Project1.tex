\documentclass[a4paper, 11pt]{report}
\usepackage{blindtext}
\usepackage[T1]{fontenc}
\usepackage[utf8]{inputenc}
\usepackage{titlesec}
\usepackage{fancyhdr}
\usepackage{geometry}

\usepackage[english]{babel}
\usepackage{apacite}

\geometry{ margin=30mm }
\counterwithin{subsection}{section}
\renewcommand\thesection{\arabic{section}.}
\renewcommand\thesubsection{\thesection\arabic{subsection}.}
\usepackage{tocloft}
\renewcommand{\cftchapleader}{\cftdotfill{\cftdotsep}}
\renewcommand{\cftsecleader}{\cftdotfill{\cftdotsep}}
\setlength{\cftsecindent}{2.2em}
\setlength{\cftsubsecindent}{4.2em}
\setlength{\cftsecnumwidth}{2em}
\setlength{\cftsubsecnumwidth}{2.5em}


\begin{document}
\titleformat{\section}
{\normalfont\fontsize{15}{0}\bfseries}{\thesection}{1em}{}
\titlespacing{\section}{0cm}{0.5cm}{0.15cm}
\titleformat{\subsection}
{\normalfont\fontsize{13}{0}\bfseries}{\thesubsection}{0.5em}{}
\titlespacing{\section}{0cm}{0.5cm}{0.15cm}

%=======================================================================================

\begin{titlepage}
\center 
\textbf{\huge INFO1111: Computing 1A Professionalism}\\[0.75cm]
\textbf{\huge 2022 Semester 1}\\[2cm]
\textbf{\huge Team Project Report}\\[3cm]

\textbf{\huge Submission number: 1}\\[0.75cm]
\textbf{\huge Team Members:}\\[0.75cm]
\textbf{\large
    \begin{tabular}{|p{0.5\textwidth}|p{0.3\textwidth}|p{0.2\textwidth}|}
        \hline
        Name & Student ID & Levels being attempted in this submission\\
        \hline
        Isaac Kim & 510603294 & Task 1, 2 \\
        Jack & ?? & ?? \\
        Kyle & ?? & ?? \\
        \hline
    \end{tabular}
}\\[0.75cm]
\end{titlepage}

%=======================================================================================
\tableofcontents
\newpage
%=======================================================================================
\section{Level 1: Basic Skills}

\subsection{Developing industry skills}
    This section is completed as a team.\\
    Throughout your Computing degree we will help you learn a range of new skills. Once you graduate however you will need to continue to learn new languages, new tools, new applications, etc. For this section you need to identify 5 approaches you can take to this continual learning. You should then put these in order from most effective to least effective, and then explain the circumstances in which each approach might be appropriate. (Target = $\sim$100 words per skill = $\sim$500 words total).

%\subsection{Skills: Nobody : Computer Science}
 %   This section will be kept blank because our group only has three members.

\subsection{Skills: Isaac Kim : Data Science}
    \subsubsection{1. Statistics}
    Statistics may look to first-year university students like myself to be a series of tedious calculations using fixed formulas. After all, that's what we've been doing in "Statistics" classes since high school. As a result, it's mind-boggling to grasp how statistical analysis and probability have such a profound impact on our daily lives. Statistics are used to forecast weather, restock store shelves, assess the state of the economy, and a variety of other things. Statistics on its own is a powerful tool for getting useful insights, but when combined with computer algorithms, machine learning, or AI, it can investigate a far larger pool of data and provide answers to business, science, and societal concerns that were previously deemed unsolvable.
    \subsubsection{2. Programming Knowldege}
    Data science necessitates a mix of math and programming skills because it sits at the intersection of analytics and engineering. Data scientists will be unable to distinguish themselves from traditional statisticians if they are unable to use programming tools. Real data scientists will use sophisticated languages like Python or R to gain access to a wide range of useful functionality for dealing with Math, Statistics, and Scientific operations. Data scientists can utilise computer programmes to apply statistical expertise in a quick and secure manner, allowing them to uncover patterns in large amounts of disorganised data and make use of them.
    \subsubsection{3. Data Wrangling}
    Data wrangling is the process of organising and transforming new data into the appropriate format for later analysis. It's a crucial step that's well worth the investment because it helps analysts make better data-driven judgments in less time. Outlier treatment, missing value imputation, scaling, and correcting data types can help make a data collection more orderly and aid in the development of more accurate statistical results. Data wrangling, in my opinion, is just as significant as Oxford Dictionary's decision to structure its print dictionary alphabetically. It couldn't be worse if an English learner had to go through 1000 pages of paper simply to discover a definition of one word. 
    \subsubsection{4. Data Visualisation}
    It might be difficult to grasp what a data collection has to offer as a story when looking at a series of statistics or numerical summaries. This is where Data Visualization, the more entertaining side of data science, may be able to assist with the solution. Data visualisation enables data scientists to convey complex data sets to scholars and the general public in a more appealing and intelligible manner. Furthermore, because computer software can automate the process, creating complex histograms or pie charts takes only a few seconds and does not require the assistance of a skilled artist.
    

\subsection{Skills: add student 3 name here : Software Development}
Your text goes here

\subsection{Skills: add student 4 name here : Cyber Security}

Your text goes here


%=======================================================================================

\newpage
\section{Level 2: Basic Technology}

Level 2 focuses on initial evaluation of the tech stack that is used by a selected company. All companies make use of a range of technologies, and these technologies need to work together. A tech stack is basically just this collection of technologies that collectively enable a company's systems. As an example, one of the most common technology stacks for supporting web servers is LAMP: Linux as the underlying operating system; Apache as a web server; MySQL as the supporting database; and Perl (or more recently PHP or Python) as the programming language.

Each student should choose a different tech stack and explain the role of each of the different technologies in that stack. Note that prior to researching your proposed tech stack and spending time writing about it, it might be a good idea to check with your tutor as to whether your chosen stack is suitable. (Target = $\sim$200-400 words per student).

\subsection{Tech Stack: add student 1 name here}

Your text goes here

\subsection{Tech Stack: add student 2 name here}

Your text goes here

\subsection{Tech Stack: add student 3 name here}

Your text goes here

\subsection{Tech Stack: add student 4 name here}

Your text goes here


%=======================================================================================

\newpage
\section{Level 3: Advanced Skills}

Level 3 focuses on more advanced technical skills (\LaTeX\ and Git) and analysis of linkages and relationships between the items in the company tech stack.

The following is a list of advanced Git and \LaTeX\ skills/features. Each student should select one pair of items from each list and demonstrate actual use of each item (either through activity in Git, or through including items in this report). (Target = $\sim$100 words per student for each feature).
\begin{itemize}
    \item Git
    \begin{itemize}
        \item Rebasing and Ignoring files
        \item Forking and Special files
        \item Resetting and Tags
        \item Reverting and Automated merges
        \item Hooks and Tags
    \end{itemize}
    \item \LaTeX\ 
    \begin{itemize}
        \item Cross-referencing and Custom commands
        \item Footnotes/margin notes and creating new environments
        \item Floating figures and editing style sheets
        \item Graphics and advanced mathematical equations
        \item Macros and hyperlinks
    \end{itemize}
\end{itemize}

\subsection{Advanced features: add student 1 name here}

Explain your use of the advanced Git and \LaTeX\ features. 

\subsection{Advanced features: add student 2 name here}

Explain your use of the advanced Git and \LaTeX\ features. 

\subsection{Advanced features: add student 3 name here}

Explain your use of the advanced Git and \LaTeX\ features. 

\subsection{Advanced features: add student 4 name here}

Explain your use of the advanced Git and \LaTeX\ features. 



%=======================================================================================

\newpage
\section{Level 4: Advanced Knowledge}

Level 4 focuses on analysing your particular tech stack and considering alternatives. Each student should consider the tech stack they described for Level 2, and then discuss each of the following points:
\begin{itemize}
    \item What are the strengths and limitations of this stack? (Target = $\sim$200 words).
    \item What alternatives exist, and under what situations might these alternatives be a better choice? (Target = $\sim$200 words).
\end{itemize}

\subsection{Advanced Knowledge: add student 1 name here}

Your text goes here

\subsection{Advanced Knowledge: add student 2 name here}

Your text goes here

\subsection{Advanced Knowledge: add student 3 name here}

Your text goes here

\subsection{Advanced Knowledge: add student 4 name here}

Your text goes here



%=======================================================================================

\newpage

\bibliographystyle{apacite}
\bibliography{main}

\end{document}
\end{report}
